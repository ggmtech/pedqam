% Options for packages loaded elsewhere
\PassOptionsToPackage{unicode}{hyperref}
\PassOptionsToPackage{hyphens}{url}
%
\documentclass[
  10pt,
  ignorenonframetext,
  aspectratio=43,
]{beamer}
\usepackage{pgfpages}
\setbeamertemplate{caption}[numbered]
\setbeamertemplate{caption label separator}{: }
\setbeamercolor{caption name}{fg=normal text.fg}
\beamertemplatenavigationsymbolsempty
% Prevent slide breaks in the middle of a paragraph
\widowpenalties 1 10000
\raggedbottom
\setbeamertemplate{part page}{
  \centering
  \begin{beamercolorbox}[sep=16pt,center]{part title}
    \usebeamerfont{part title}\insertpart\par
  \end{beamercolorbox}
}
\setbeamertemplate{section page}{
  \centering
  \begin{beamercolorbox}[sep=12pt,center]{part title}
    \usebeamerfont{section title}\insertsection\par
  \end{beamercolorbox}
}
\setbeamertemplate{subsection page}{
  \centering
  \begin{beamercolorbox}[sep=8pt,center]{part title}
    \usebeamerfont{subsection title}\insertsubsection\par
  \end{beamercolorbox}
}
\AtBeginPart{
  \frame{\partpage}
}
\AtBeginSection{
  \ifbibliography
  \else
    \frame{\sectionpage}
  \fi
}
\AtBeginSubsection{
  \frame{\subsectionpage}
}
\usepackage{amsmath,amssymb}
\usepackage{lmodern}
\usepackage{iftex}
\ifPDFTeX
  \usepackage[T1]{fontenc}
  \usepackage[utf8]{inputenc}
  \usepackage{textcomp} % provide euro and other symbols
\else % if luatex or xetex
  \ifXeTeX
    \usepackage{mathspec}
  \else
    \usepackage{unicode-math}
  \fi
  \defaultfontfeatures{Scale=MatchLowercase}
  \defaultfontfeatures[\rmfamily]{Ligatures=TeX,Scale=1}
\fi
\usetheme[]{PaloAlto}
\usecolortheme{dolphin}
\usefonttheme{structurebold}
% Use upquote if available, for straight quotes in verbatim environments
\IfFileExists{upquote.sty}{\usepackage{upquote}}{}
\IfFileExists{microtype.sty}{% use microtype if available
  \usepackage[]{microtype}
  \UseMicrotypeSet[protrusion]{basicmath} % disable protrusion for tt fonts
}{}
\makeatletter
\@ifundefined{KOMAClassName}{% if non-KOMA class
  \IfFileExists{parskip.sty}{%
    \usepackage{parskip}
  }{% else
    \setlength{\parindent}{0pt}
    \setlength{\parskip}{6pt plus 2pt minus 1pt}}
}{% if KOMA class
  \KOMAoptions{parskip=half}}
\makeatother
\usepackage{xcolor}
\newif\ifbibliography
\setlength{\emergencystretch}{3em} % prevent overfull lines
\providecommand{\tightlist}{%
  \setlength{\itemsep}{0pt}\setlength{\parskip}{0pt}}
\setcounter{secnumdepth}{-\maxdimen} % remove section numbering
\usepackage{fancyvrb}
\ifLuaTeX
  \usepackage{selnolig}  % disable illegal ligatures
\fi
\IfFileExists{bookmark.sty}{\usepackage{bookmark}}{\usepackage{hyperref}}
\IfFileExists{xurl.sty}{\usepackage{xurl}}{} % add URL line breaks if available
\urlstyle{same} % disable monospaced font for URLs
\hypersetup{
  pdftitle={Presentation on Vendor Management},
  pdfauthor={Gopal Kumar, PED/RS},
  hidelinks,
  pdfcreator={LaTeX via pandoc}}

\title{Presentation on Vendor Management}
\subtitle{Vigilance awareness week lecture}
\author{Gopal Kumar, PED/RS}
\date{2022-11-02}
\institute{RDSO, Lucknow}

\begin{document}
\frame{\titlepage}

\begin{frame}[allowframebreaks]
  \tableofcontents[hideallsubsections]
\end{frame}
\hypertarget{introduction}{%
\section{Introduction}\label{introduction}}

\begin{frame}{What is Vendor Management}
\protect\hypertarget{what-is-vendor-management}{}
\begin{block}{Make or buy decision}
\protect\hypertarget{make-or-buy-decision}{}
\begin{itemize}
\tightlist
\item
  We outsource certain products or services as partner vendors can offer
  better service/product or a better economy.
\item
  Businesses rarely have all the resources to execute projects and
  fulfill business objectives on their own.
\end{itemize}
\end{block}

\begin{block}{Vendor is a partner}
\protect\hypertarget{vendor-is-a-partner}{}
\begin{itemize}
\tightlist
\item
  Vendor gives better technology or service /cost economies
\item
  must be nurtured
\item
  within the Government rules of constitutional equality among
  persons/business entities
\end{itemize}
\end{block}
\end{frame}

\begin{frame}{Vendor management : conception to disposal}
\protect\hypertarget{vendor-management-conception-to-disposal}{}
\begin{block}{Vendor Management}
\protect\hypertarget{vendor-management}{}
\begin{itemize}
\tightlist
\item
  Complete process of acquiring and managing suppliers
\item
  with different points of contact, rates, and contract terms
\item
  in complex and highly dynamic buyer-vendor ecosystem.
\end{itemize}
\end{block}

\begin{block}{General key points in Vendor selection}
\protect\hypertarget{general-key-points-in-vendor-selection}{}
\begin{itemize}
\tightlist
\item
  Select vendors that present greatest value at the least viable
  expenditure.
\item
  Mission-critical vendor associate for their concerns and relationship.
\item
  Building long term trust and mutual support is Important
\item
  Needs to fit into overall delivery of service and corporate strategy
\item
  Understand vendor's business model as bottom line for Vendor is and
  important and cost and delivery pressure on them will likely see a
  commensurate reduction in service/product quality.
\end{itemize}
\end{block}
\end{frame}

\hypertarget{vendor-development-in-railways}{%
\section{Vendor development in
Railways}\label{vendor-development-in-railways}}

\begin{frame}{Vendor development in Railways}
\begin{block}{Vendor development in Railways}
\protect\hypertarget{vendor-development-in-railways-1}{}
\begin{itemize}
\tightlist
\item
  Most aspect of vendor management handled by Stores department
\item
  Item demands, technical requirements and usage feedback performance
  report by user departments
\end{itemize}
\end{block}

\begin{block}{For safety critical items - vendor pre approval}
\protect\hypertarget{for-safety-critical-items---vendor-pre-approval}{}
\begin{itemize}
\tightlist
\item
  For safety and critical items, vendors are pre-approved by RDSO/PU
\item
  RDSO approve vendors for for items assigned by Railway Board
\item
  RDSO vendor development and management is only a limited part of
  entire vendor management
\end{itemize}
\end{block}
\end{frame}

\hypertarget{rdso-vendor-development-system}{%
\section{RDSO vendor development
system}\label{rdso-vendor-development-system}}

\begin{frame}[fragile]{RDSO vendor development system}
The presentation will cover few key points of
\texttt{RDSO\ vendor\ registration\ process}

\begin{block}{RDSO vendor management activities}
\protect\hypertarget{rdso-vendor-management-activities}{}
\begin{enumerate}
\tightlist
\item
  Item identified and assigned by Railway Board
\item
  RDSO develop Specification including STR and indicative costing
\item
  RDSO Solicit vendors interest
\item
  Entire Vendor Application processing on RDSO UVAM portal at
  ireps.gov.in
\item
  Undertake Inspection of classified Category-II critical items
\item
  Evaluate Vendor performance for up-gradation and corrective actions
\item
  Failure reporting monitoring and regular and special audits
\end{enumerate}
\end{block}
\end{frame}

\begin{frame}{RDSO Vendor Development process and monitoring}
\protect\hypertarget{rdso-vendor-development-process-and-monitoring}{}
\begin{block}{Specification development}
\protect\hypertarget{specification-development}{}
\begin{itemize}
\item
  RDSO follow the basic principles of WTO-TBT `Code of Good Practice for
  the Preparation, Adoption and Application of Standards'.
\item
  The six principles of standardisation:

  \begin{enumerate}
  [a)]
  \tightlist
  \item
    Transparency
  \item
    Openness
  \item
    Impartiality and Consensus
  \item
    Effectiveness and Relevance
  \item
    Coherence
  \item
    Development Dimension
  \end{enumerate}
\end{itemize}
\end{block}
\end{frame}

\begin{frame}{Standards}
\protect\hypertarget{standards}{}
\begin{block}{What are Standards}
\protect\hypertarget{what-are-standards}{}
\begin{quote}
Document established by consensus amongst various stakeholders and
provides for common and repeated use, rules, guidelines or
characteristics for specific products or services and is aimed at
achieving optimum degree of order in a given context. These could be
specifications, procedures and guidelines to ensure that products,
services and systems are safe, consistent and reliable.
\end{quote}
\end{block}

\begin{block}{Development of Specifications}
\protect\hypertarget{development-of-specifications}{}
\begin{itemize}
\tightlist
\item
  Product conceptualisation\\
\item
  Overall functional requirement specifications of the product
\item
  Who can make it best - ourself or other person (vendor)
\item
  Mostly, Vendor specifications made in consultation with stakeholders
  for optimum cost and functionality
\end{itemize}
\end{block}
\end{frame}

\begin{frame}{Standards development process of RDSO}
\protect\hypertarget{standards-development-process-of-rdso}{}
\begin{block}{Justificatons for New specifications/STR}
\protect\hypertarget{justificatons-for-new-specificationsstr}{}
\begin{enumerate}
[a)]
\tightlist
\item
  Why the need for creation of new Standard has arisen?
\item
  What is the existing Vendor base and the action plan to develop new
  vendors?
\item
  What will be the Pricing of the item? (Some indicative estimation of
  pricing of the item shall be indicated in note for information)
\end{enumerate}
\end{block}

\begin{block}{Changes in the Existing RDSO Standards}
\protect\hypertarget{changes-in-the-existing-rdso-standards}{}
Changes of raw material / process / testing /procedures / guidelines or
validation of product or services.

\begin{enumerate}
[a)]
\tightlist
\item
  Why the need for Change has arisen?
\item
  parameters likely to be affected?
\item
  Effect of change on the existing vendor base?
\item
  Effect of the change on the pricing of the item.
\item
  Effect on service life of component/equipment.
\item
  Mechanisms for effective monitoring those parameters
\item
  Any other remarks
\end{enumerate}
\end{block}
\end{frame}

\hypertarget{rdso-vendor-registraion-basics-and-process}{%
\section{RDSO Vendor registraion basics and
process}\label{rdso-vendor-registraion-basics-and-process}}

\begin{frame}{RDSO Vendor registraion basics and process}
\begin{itemize}
\tightlist
\item
  RDSO only register OEMs
\item
  No Traders or resellers permitted.
\item
  Must have ISO 9000 certification for manufacturing the item classes
\item
  Overseas OEM's may appoint Indian agent but approval to OEM only.
\end{itemize}

Registration brief follows -\textgreater{}
\end{frame}

\begin{frame}{Prerequisite for application}
\protect\hypertarget{prerequisite-for-application}{}
\begin{block}{RDSO Prerequisite for application (ISO Para 4.3 )}
\protect\hypertarget{rdso-prerequisite-for-application-iso-para-4.3}{}
All applicants shall possess the following pre-requisites

\begin{itemize}
\tightlist
\item
  ISO9001 Certification (ISO para 4.3.1 )for manufacture of same/similar
  item at his works address
\item
  ISO certificate agency accreditation verified at www.iaf.nu
\item
  Digital Signatures (ISO para 4.3.2 but optional? )
\end{itemize}
\end{block}
\end{frame}

\begin{frame}{Mandatory Declarations by Vendors (ISO para 4.2 )}
\protect\hypertarget{mandatory-declarations-by-vendors-iso-para-4.2}{}
\begin{itemize}
\item
  Document as per ISO apex QO-F-8.1-7 `List of documents to be sought
  from the vendor at the time of fresh registration and Annexures/Forms'
\item
  Declaration for classifying Allied / Sister Concerns Declaration shall
  be submitted in the format as per Annexure A-4 \& Annexures-1\&2 of
  document QO-F-8.1-7 (ISO 4.2.1)
\item
  Undertaking to be submitted by the applicant in case of change of
  status disclose to RDSO, any changes in the status i.e.~name, address,
  work place etc., ( Annexure A-3 of document QO-F-8.1-7 ) (ISO 4.2.2 )
\item
  Undertaking for not changing M \& P (Annexure A-3 of document no.
  QO-F-8.1- 7) without intimation to RDSO (fax/email even for repairs)
  (ISO 4.2.3 )
\end{itemize}
\end{frame}

\begin{frame}{Mandatory Declarations by Vendors (ISO para 4.2 ) cont..}
\protect\hypertarget{mandatory-declarations-by-vendors-iso-para-4.2-cont..}{}
\begin{itemize}
\item
  documents digitally signed by the authorized representative plus
  authorization letter (as per Ann A-5 of document no. QO-F-8.1-7) from
  firms authorised signatory ( Director/Proprietor/Partner on behalf of
  the firm.
\item
  Charges paid by all on UVAM Module of IREPS website.
\end{itemize}
\end{frame}

\begin{frame}{Registration of Foreign Firms}
\protect\hypertarget{registration-of-foreign-firms}{}
\begin{itemize}
\item
  Registration of Foreign Firms ( Para 4.2.4)
\item
  Focus on Make in India
\end{itemize}
\end{frame}

\hypertarget{vendor-application-processing}{%
\section{Vendor Application
Processing}\label{vendor-application-processing}}

\begin{frame}{Vendor Application Processing}
\begin{itemize}
\item
  QO-D-8.1-6 Vendor Application Processing
\item
  Only for items allotted to RDSO by Railway Board/ D.G., RDSO
\item
  Safekeeping of documents with IPR's (para 4.1 ) down loaded. ( Hard
  copies can be kept in separate files by the respective Directorate.
  (Is it required with UVAM?)
\end{itemize}
\end{frame}

\begin{frame}{Entity change while applicaition processing}
\protect\hypertarget{entity-change-while-applicaition-processing}{}
(ISO para - 4.2 ) Not allowed but if unavoidable, for change of
Ownership with name/without name, Merger, Take Over, Acquisition, Major
Changes in Share Holding/ Directors of company, change in type of firm
from Proprietorship/Partnership/Pvt. Limited etc. the same application
can be processed.

\begin{itemize}
\item
  no change in Work Address, Machinery \& Plant, Bill of Material,
  Process defined in QAP etc. affecting the quality of product.
\item
  All revised declarations of QO-F-8.1-7 ``List of documents to be
  sought from the vendor'' taken
\item
  For complex changes, firm to apply fresh.
\item
  Change of address of plant/works allowed till no visit made by RDSO to
  vendor's works place.
\end{itemize}
\end{frame}

\hypertarget{application-screening-clarifications}{%
\section{Application Screening \&
clarifications}\label{application-screening-clarifications}}

\begin{frame}{Application Screening \& clarifications (ISO para 4.3)}
\protect\hypertarget{application-screening-clarifications-iso-para-4.3}{}
\begin{block}{Scrutinized in detail for}
\protect\hypertarget{scrutinized-in-detail-for}{}
\begin{itemize}
\tightlist
\item
  adequacy in respect of the information sought.
\item
  Documents signed by an authorized signatory.
\item
  Inadequacy pin pointed and communicated to firm on line/ email and
  Vendor to submit compliance online.
\end{itemize}
\end{block}
\end{frame}

\begin{frame}{Screening for Sister / Allied concern (para 4.3.1)}
\protect\hypertarget{screening-for-sister-allied-concern-para-4.3.1}{}
The self-declaration received vide rec-ref-1 QO-F-8.1-7 be examined

\begin{block}{Definition of Allied / Sister concern}
\protect\hypertarget{definition-of-allied-sister-concern}{}
\begin{itemize}
\tightlist
\item
  Section 40A(2)(b) of Income-Tax Act and Section 370-1(B) of Companies'
  Act and further modified to suit our requirements is as under -
\item
  4.3.1.1 Proprietary Firms - All the firms owned by same person are
  allied / sister concerns.
\item
  4.3.1.2 Partnership Firms - All firms having the same set of partners.
  If one or more partners with over 20\% profit sharing ratio of 20\%
\item
  4.3.1.3 Companies -
\item
  Companies having -- ``majority'' of Directors common
\item
  Common shareholder having 1/3rd shares or more.
\item
  One or more Directors or close relatives (degree defined) shareholding
  over 1/3rd
\item
  4.3.1.4 Other Conditions - firms/companies operating from same office
  or same works.
\end{itemize}
\end{block}
\end{frame}

\begin{frame}
\begin{block}{Undertaking for legal formalities and statutory
compliances for vendor registration}
\protect\hypertarget{undertaking-for-legal-formalities-and-statutory-compliances-for-vendor-registration}{}
\begin{itemize}
\item
  Undertaking as per Annexure A-3 of document QO-F-8.1-7 abiding with
  all legal formalities and statutory compliances and produce when
  demanded. (Para 4.3.2.1)
\item
  Deficiency/ non-compliance at any stage can result delisting even
  while the firm is listed on the RDSO's vendor list.
\end{itemize}
\end{block}
\end{frame}

\hypertarget{technical-screening-para-4.3.3-of-iso}{%
\section{Technical Screening (Para 4.3.3 of
ISO)}\label{technical-screening-para-4.3.3-of-iso}}

\begin{frame}{Technical Screening (Para 4.3.3 of ISO)}
\begin{block}{Technical Screening}
\protect\hypertarget{technical-screening}{}
\begin{itemize}
\item
  A technical screening on the basis of information regarding
  infrastructure \& manufacturing practices, QAP etc., firm can be
  considered for a visit for CCA
\item
  Outsourcing of some minor activities at ED/PED and incorporated in QAP
  and got to be verified during assessment of the firm.
\item
  Outsourcing to any sister concern permitted, subject to compliance of
  specs/ STR/ ISO etc. by the sister concern. ( covered in QAP).\\
\item
  Non-compliance of any issue in sister concern, action against approved
  vendor.
\item
  RDSO officials shall have to visit all outsourced work-places of
  sister concern to certify compliance before approval
\item
  For overseas firms and their workplaces, Spl. DG/VD's dispensation can
  be taken on case to case basis giving justification.
\end{itemize}
\end{block}
\end{frame}

\begin{frame}{Manufacturing facilies at multiple location and
outsourcing}
\protect\hypertarget{manufacturing-facilies-at-multiple-location-and-outsourcing}{}
\begin{block}{Manufacturing facilies at multiple location}
\protect\hypertarget{manufacturing-facilies-at-multiple-location}{}
Preferably manufacturing and testing facilities covered under STR at one
premises. Facilities spread in more than one place provided they are
under the same ownership with same name and activities are clearly spelt
out in the QAP. ( termed as Ancillary units). (Para 4.3.3.3 )
\end{block}
\end{frame}

\begin{frame}{Acceptance of application for CCA}
\protect\hypertarget{acceptance-of-application-for-cca}{}
\begin{itemize}
\tightlist
\item
  Acceptance of application after scrutiny of the documents (status on
  the UVAM Module) (Para 4.4 )
\item
  Capacity/ capability assessment even if minor deficiencies but
  deficiencies have to be set right before approval (CCA).
\item
  Major deficiencies to be uploaded on UVAM firm will get one month time
  to make good the deficiencies.
\item
  On Merit may be permitted maximum of three Months before the closure
\item
  Nevertheless, know that directorate head responsible for all delays
  byeond 2 months
\end{itemize}
\end{frame}

\begin{frame}{Rejection of the Application}
\protect\hypertarget{rejection-of-the-application}{}
\begin{itemize}
\item
  Rejection of the Application shall be informed through IREPS, UVAM
  Module about the deficiencies. The firm will have to apply afresh.
  (Para 4.5 )
\item
  In case approved vendors \textless5, no registration fee if
  reapplication done within six months ( a onetime exception).
\end{itemize}
\end{frame}

\begin{frame}{Physical visit to firm's permises for CCA}
\protect\hypertarget{physical-visit-to-firms-permises-for-cca}{}
\begin{itemize}
\item
  Visit to firm's premises by RDSO official for STR and CCA
\item
  Digitally signed submitted documents be verified with originals.
\item
  Legal documents duly verified and all original taken to RDSO.
\item
  If visit waived off, then RDSO to collect documents from firm
\item
  Only two postponements else case be closed
\item
  Alternative Vendor remote inspection feasibility for STR /CCA
  verification by remote means can also be used.
\item
  For overseas suppliers - through RA Berlin/DRA Paris/London or other
  IR official abroad. Spl. DG/VD may approve case basis specific
  arrangement
\end{itemize}
\end{frame}

\hypertarget{capacity-cum-capability-assessment-cca-report}{%
\section{Capacity cum Capability Assessment (CCA)
Report}\label{capacity-cum-capability-assessment-cca-report}}

\begin{frame}{CCA Report}
\protect\hypertarget{cca-report}{}
\begin{itemize}
\item
  CCA Report in standard format (document no. QO-F-8.1-8) in digitally
  signed pdf through UVAM with his remarks (Para 4.7 )
\item
  Once deficiency found during CCA, case be CLOSED through UVAM.
\item
  Re Registration within 6 months, the case processed from the stage
  where it was closed (as a onetime exception)
\end{itemize}
\end{frame}

\begin{frame}{Approval by the ED/PED controlling the item (Para 4.8 )}
\protect\hypertarget{approval-by-the-edped-controlling-the-item-para-4.8}{}
On Successful verification for documents and CCA such Firms placed as
Developmental Vendors for 24 months, with a check note subjected to: 1.
Technical Clearance of Test Samples/Prototype by the RDSO. 2. Successful
Field Trials of the Specified Quantity

\begin{block}{Developmental Vendors Orders quantity}
\protect\hypertarget{developmental-vendors-orders-quantity}{}
\begin{itemize}
\tightlist
\item
  Developmental Vendors item purchase ( Board's letter
  99/RS(G)/709/1/Pt. dated 13.01.2015 para (3 A ii) ) permits qty.
  ordered on developmental vendors in and outside the NPQ
\item
  Board's letter No: 2001/ RS(G)/779/7/Pt.2 dated 06.11.2018. Before
  initiating the regular supplies - Technical Clearance of
  Prototype/Test sample and successful field trials as specified for
  item by RDSO is to be completed.
\end{itemize}
\end{block}
\end{frame}

\begin{frame}
\begin{block}{Vendor initiative sample/prototype testing}
\protect\hypertarget{vendor-initiative-sampleprototype-testing}{}
\begin{itemize}
\item
  Vendors can approach to the RDSO for sample/prototype testing,
  irrespective of whether the vendor gets the order from Railway units
  or not in order to expedite the approval process.
\item
  Check notes removed after Technical clearance of Test Samples/
  Prototype/ Successful Field Trials of the Specified Quantity
\item
  If Technical clearance and field trials takes more than 24 months, ED
  to decide whether to continue the firm in the list of Developmental
  Vendor with check note as per procedure.
\item
  If Vendor gets orders and supply done/under process, the extra
  validity upto two years.
\end{itemize}
\end{block}
\end{frame}

\begin{frame}{QAP}
\protect\hypertarget{qap}{}
\begin{block}{Approved and digitally signed copy of QAP}
\protect\hypertarget{approved-and-digitally-signed-copy-of-qap}{}
4.12 Approved and digitally signed copy of QAP with work address for the
approved product to be sent to the vendor for implementation, duly
stamped as `approved'.
\end{block}

\begin{block}{Review / Up-gradation of QAP}
\protect\hypertarget{review-up-gradation-of-qap}{}
4.13 Review / Up-gradation of QAP by \textgreater{} Jt. Director of the
field unit in the light of fresh technical information received from
Design Directorate \& Information Received after field trial from the
Users.
\end{block}
\end{frame}

\hypertarget{approval-of-vendor}{%
\section{Approval of Vendor}\label{approval-of-vendor}}

\begin{frame}{Communication of Vendor approval throuth UVAM}
\protect\hypertarget{communication-of-vendor-approval-throuth-uvam}{}
\begin{itemize}
\tightlist
\item
  Communication of approval through a letter generated through `UVAM''
  and status of automatically updated in respective `List of Vendors for
  Development Orders'/ `List of Approved Vendors' on 'UVAM'' portal.
  (4.14)
\end{itemize}
\end{frame}

\begin{frame}{Relaxation of Vendor registration requirements (QO-D-8.1-7
)}
\protect\hypertarget{relaxation-of-vendor-registration-requirements-qo-d-8.1-7}{}
\begin{itemize}
\item
  4.1 Relaxation for vendors with Railway PU Item transferred from PU
  shall remain same but undertaking in. QO-F-8.1-7 be taken
\item
  Relaxation as provided under product development directive in (ref-11)
  (see ISO 4.2 )
\item
  Procedure for approval of firms/vendors given IPR by the principal IPR
  holder for manufacture in India as stated in para 4.3
\item
  Firms/vendors holding IPR of a proven product or firms/vendors duly
  authorized to use the same by the principal IPR holder (for
  manufacture in India inducted into the Approved List after capacity,
  capability assessment including compliance to QAP \& STR for the given
  product. (4.11.1.2 of QO-D-8.1-6.)
\end{itemize}
\end{frame}

\begin{frame}{Cross approvals}
\protect\hypertarget{cross-approvals}{}
\begin{itemize}
\item
  Applicants registered with reputed units like CPSU/Metro/Power Grid /
  Ordinance Factories / DFCCL, for same item be done. (Para 4.4)
\item
  Para-wise compliance of specification and deviation if any with
  mitigation measure/ alternate provision.
\item
  May waive prototype requirements and/or trial requirements in addition
  to CCA and the vendor can be directly placed in developmental vendor
  category.
\item
  In deserving cases, after recording reasons, especially if developed
  vendors are less than three and such vendor considered capable based
  on satisfactory supply performance in such PSUs.
\item
  Cross approval policy may be followed where it exists/applicable.
\end{itemize}
\end{frame}

\hypertarget{quality-audit-of-approved-vendor}{%
\section{Quality Audit of Approved
Vendor}\label{quality-audit-of-approved-vendor}}

\begin{frame}{Quality Audit of Approved Vendor}
\begin{block}{QO-D-8.1-13 Quality Audit of Approved Vendor}
\protect\hypertarget{qo-d-8.1-13-quality-audit-of-approved-vendor}{}
\begin{itemize}
\item
  To ensure the quality of the material supplied by the firms
  checks/Quality audit on their QAM including M\&P, man-power, sources
  and their own internal quality checks to ensure they are in place as
  per conditions laid down while approving the firm.
\item
  4.1 Periodicity between 3 to 5 years from last quality audit. Do it in
  3 year
\item
  Quality Audit at officer's level. ED / PED may depute supervisors with
  the approval of Spl. DG/VD citing specific reasons why quality audit
  can't be postponed
\item
  Can follow special remote procedure if accepted by ED
\item
  4.2.1 Quality Audit Format in document no. QO-F-8.1 -9 to ED
\end{itemize}
\end{block}
\end{frame}

\begin{frame}{Changes in vendor facilites}
\protect\hypertarget{changes-in-vendor-facilites}{}
\begin{itemize}
\tightlist
\item
  Vendor responsible to immediately inform full technical details of any
  changes about Bill of Material, plant \& machinery and Quality
  Assurance Plan (para 4.3 )
\item
  In case the approved vendor fails to comply with the above provision,
  his name may be deleted/temporarily withdrawn
\end{itemize}
\end{frame}

\hypertarget{vendor-monitoring-and-performance-evaluation}{%
\section{Vendor Monitoring and performance
evaluation}\label{vendor-monitoring-and-performance-evaluation}}

\begin{frame}{Performance Feedback reports from user Railways}
\protect\hypertarget{performance-feedback-reports-from-user-railways}{}
\begin{enumerate}
\tightlist
\item
  RDSO web portal hosted on railnet website 10.100.12.19
\item
  Feed back from Railways (Discouraged as subjective)
\item
  Online failure reporting work flow from Freight Maintenance Management
  (FMM)/CMM/WMS (better system to be formalised)
\end{enumerate}
\end{frame}

\begin{frame}{Evaluation of performance (para 4.4 )}
\protect\hypertarget{evaluation-of-performance-para-4.4}{}
\begin{itemize}
\tightlist
\item
  Based on feedback from user as available in the directorate during
  quality audit.
\item
  Refusal to Quality Audit Date of Quality Audit intimated to firm
  through email/fax at least 30 days in advance. (Para 4.5)
\item
  In case date not accepted, next and final date after 30 days
  intimation to firm
\item
  In case approved vendor refuses to quality audit or does not allow
  RDSO to perform quality audit, vendor temporarily delisted till
  quality audit is successfully done.
\end{itemize}
\end{frame}

\begin{frame}{Quality and performance deterioration}
\protect\hypertarget{quality-and-performance-deterioration}{}
Deterioration of performance \& out of turn Quality Audit (Para 4.6 )

\begin{itemize}
\item
  ED decide out of turn quality audit for further continuance of
  approval/ temporary delisting/ delisting.
\item
  ED/ PED may decide to stop the inspection temporarily considering
  seriousness
\item
  Audit Report communicated and vendor status be updated on website
  (Para 4.7 )
\item
  Discrepancies during Quality Audits be taken up with show cause notice
  to comply deficiencies , and implementing CAPA within 30 days. This
  shall be verified. (Para 4.8)
\item
  If the firm fails to comply, firm be temporary delisted
\end{itemize}
\end{frame}

\begin{frame}{Inspections ( Suggestions)}
\protect\hypertarget{inspections-suggestions}{}
\begin{itemize}
\item
  Inspection - Adds no values to quality, waste of resource but
  essential under mistrusted vendors
\item
  Better develop strong and trusted vendors with lasting relationships
  with suppliers.
\item
  Firms WTC may be considered depending on robustness of firms QAP and
  mitigation activities
\item
  Private sector insists on process monitoring from its manufacturers.
  Instead of checking a complete lot and then rejecting it, we need to
  look into that aspect. That brings quality and eases inspection for
  the supplier
\end{itemize}
\end{frame}

\begin{frame}{Quality key items to consider (suggestions)}
\protect\hypertarget{quality-key-items-to-consider-suggestions}{}
\begin{block}{7 QC tools - Prof.~Ishikawa, Tokyo Univ}
\protect\hypertarget{qc-tools---prof.-ishikawa-tokyo-univ}{}
Solves about \textasciitilde{} 95\% of problems

\begin{enumerate}
\tightlist
\item
  Pareto Chart
\item
  Run Chart - run-sequence plot
\item
  Histogram - underlying distribution - process capability
\item
  Cause-and-effect diagram - Ishikawa fishbone diagram : root causes
  effect into six Ms: measurement, material, machine, method, manpower
  and mother nature. Brainstorming ``why does this happen?''
\item
  Scatter plots : X-Y graph data relationship but no causation
\item
  Control charts : processes variation from specs.( x-bar , R , S , c ,
  u , np and p chart)
\item
  Check Sheets : defect concentration diagram, structured forms
\end{enumerate}

Vendors following above likely to have robust internal quality control
system
\end{block}
\end{frame}

\begin{frame}
\begin{block}{Vendor supplies performance monitoring:}
\protect\hypertarget{vendor-supplies-performance-monitoring}{}
\begin{itemize}
\tightlist
\item
  Warranty portal and in future FMM/CMM/WMS in mech deptt
\item
  Reliability analyses and life tables statistical analysis with
  censured data
\item
  Consider Total effective life cycle cost
\item
  Good vendors are assets and proved values to the business
\end{itemize}
\end{block}
\end{frame}

\begin{frame}{Few Suggestions for discussions}
\protect\hypertarget{few-suggestions-for-discussions}{}
\begin{block}{Documentation and legal compliances}
\protect\hypertarget{documentation-and-legal-compliances}{}
\begin{itemize}
\item
  Based on certificate from statutory charted accountant
\item
  All CCA compliance by chartered engineering firms
\item
  RDSO can always verify and cross check certificates
\item
  like system of Advocates in Court, CA for company accounts, CHA for
  export
\item
  Ways to incentivise vendors meeting better process capabilities (Cp)
\end{itemize}
\end{block}
\end{frame}

\begin{frame}{Thank you very much}
\protect\hypertarget{thank-you-very-much}{}
\begin{block}{Thank you very much}
\protect\hypertarget{thank-you-very-much-1}{}
Thank you
\end{block}
\end{frame}

\end{document}
