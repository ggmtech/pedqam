% Options for packages loaded elsewhere
\PassOptionsToPackage{unicode}{hyperref}
\PassOptionsToPackage{hyphens}{url}
%
\documentclass[
  ignorenonframetext,
]{beamer}
\usepackage{pgfpages}
\setbeamertemplate{caption}[numbered]
\setbeamertemplate{caption label separator}{: }
\setbeamercolor{caption name}{fg=normal text.fg}
\beamertemplatenavigationsymbolsempty
% Prevent slide breaks in the middle of a paragraph
\widowpenalties 1 10000
\raggedbottom
\setbeamertemplate{part page}{
  \centering
  \begin{beamercolorbox}[sep=16pt,center]{part title}
    \usebeamerfont{part title}\insertpart\par
  \end{beamercolorbox}
}
\setbeamertemplate{section page}{
  \centering
  \begin{beamercolorbox}[sep=12pt,center]{part title}
    \usebeamerfont{section title}\insertsection\par
  \end{beamercolorbox}
}
\setbeamertemplate{subsection page}{
  \centering
  \begin{beamercolorbox}[sep=8pt,center]{part title}
    \usebeamerfont{subsection title}\insertsubsection\par
  \end{beamercolorbox}
}
\AtBeginPart{
  \frame{\partpage}
}
\AtBeginSection{
  \ifbibliography
  \else
    \frame{\sectionpage}
  \fi
}
\AtBeginSubsection{
  \frame{\subsectionpage}
}
\usepackage{amsmath,amssymb}
\usepackage{lmodern}
\usepackage{ifxetex,ifluatex}
\ifnum 0\ifxetex 1\fi\ifluatex 1\fi=0 % if pdftex
  \usepackage[T1]{fontenc}
  \usepackage[utf8]{inputenc}
  \usepackage{textcomp} % provide euro and other symbols
\else % if luatex or xetex
  \usepackage{unicode-math}
  \defaultfontfeatures{Scale=MatchLowercase}
  \defaultfontfeatures[\rmfamily]{Ligatures=TeX,Scale=1}
\fi
% Use upquote if available, for straight quotes in verbatim environments
\IfFileExists{upquote.sty}{\usepackage{upquote}}{}
\IfFileExists{microtype.sty}{% use microtype if available
  \usepackage[]{microtype}
  \UseMicrotypeSet[protrusion]{basicmath} % disable protrusion for tt fonts
}{}
\makeatletter
\@ifundefined{KOMAClassName}{% if non-KOMA class
  \IfFileExists{parskip.sty}{%
    \usepackage{parskip}
  }{% else
    \setlength{\parindent}{0pt}
    \setlength{\parskip}{6pt plus 2pt minus 1pt}}
}{% if KOMA class
  \KOMAoptions{parskip=half}}
\makeatother
\usepackage{xcolor}
\IfFileExists{xurl.sty}{\usepackage{xurl}}{} % add URL line breaks if available
\IfFileExists{bookmark.sty}{\usepackage{bookmark}}{\usepackage{hyperref}}
\hypersetup{
  pdftitle={The Diesel Loco Performance},
  pdfauthor={Gopal Kumar\^{}1},
  hidelinks,
  pdfcreator={LaTeX via pandoc}}
\urlstyle{same} % disable monospaced font for URLs
\newif\ifbibliography
\usepackage{longtable,booktabs,array}
\usepackage{calc} % for calculating minipage widths
\usepackage{caption}
% Make caption package work with longtable
\makeatletter
\def\fnum@table{\tablename~\thetable}
\makeatother
\setlength{\emergencystretch}{3em} % prevent overfull lines
\providecommand{\tightlist}{%
  \setlength{\itemsep}{0pt}\setlength{\parskip}{0pt}}
\setcounter{secnumdepth}{-\maxdimen} % remove section numbering
\usepackage{xcolor}
\usepackage{framed}
\ifluatex
  \usepackage{selnolig}  % disable illegal ligatures
\fi

\title{The Diesel Loco Performance}
\author{Gopal Kumar\(^1\)}
\date{2021-10-31}
\institute{\(^1\)CMPE(Diesel)}

\begin{document}
\frame{\titlepage}

\begin{frame}
\#
\end{frame}

\begin{frame}{The meeting agenda}
\protect\hypertarget{the-meeting-agenda}{}
\begin{itemize}[<+->]
\tightlist
\item
  Discussion on Compliance of last meeting Minutes
\end{itemize}

\begin{block}{contd}
\protect\hypertarget{contd}{}
\begin{itemize}[<+->]
\tightlist
\item
  Preparedness of winter Drive.
\item
  Innovation \& good work done in sheds during last Six months.
\item
  Implementation of Action plan targets for 2017-18 including plan
  issued by Railway Board
\end{itemize}
\end{block}
\end{frame}

\begin{frame}{Performance Statistics}
\protect\hypertarget{performance-statistics}{}
\colorlet{shadecolor}{red!30} 
\begin{shaded}
Black text with a red background.
\end{shaded}

\colorlet{shadecolor}{red!60}

\begin{block}{Direct PL /100ML/Month Apr-Sept}
\protect\hypertarget{direct-pl-100mlmonth-apr-sept}{}
\begin{longtable}[]{@{}lrrr@{}}
\caption{Direct PL /100ML/Month Apr-Sept.}\tabularnewline
\toprule
Shed & 2016 & 2017 & \% improvement \\
\midrule
\endfirsthead
\toprule
Shed & 2016 & 2017 & \% improvement \\
\midrule
\endhead
TKD & 12.6 & 7.2 & 42.9 \\
LDH & 8.0 & 5.5 & 31.8 \\
LKO & 17.9 & 12.3 & 31.3 \\
NR & 12.2 & 8.0 & 34.4 \\
IR & 9.2 & 7.7 & 16.7 \\
\bottomrule
\end{longtable}

\colorlet{shadecolor}{red!90}

\begin{shaded}

Black text with a darker red background.

\end{shaded}

QO-D-8.1-11 Vendor - Changes in approved status

4.1.1 Prerequisites

\begin{enumerate}[<+->]
[a)]
\item
  The vendor has applied in writing to RDSO for upgradation from ``List
  of RDSO Vendors for developmental order‟ posted on RDSO website to
  ``List of Approved Vendors‟.
\item
  The Vendor should meet either criteria i) or criteria ii) below:
\end{enumerate}

Criteria i) The vendor should have supplied minimum specified quantity
(N) of material as specified by the concerned Design/QA directorate as a
''List of RDSO Vendors for developmental order‟ posted on RDSO website
vendor To be in service for a minimum period of one year. or 15 months
from the date of issue of last inspection certificate.

12/15 months from time N Qty completed while him being dev vendor

Criteria ii) ``Equipment Months‟ on basis of in service period can also
be considered as qualifying criteria for upgradation.

If ``N‟ is the minimum specified quantity then minimum Equipment Months
for upgradation from date of in service = 12 N or Minimum Equipment
Months for up-gradation from date of issue of inspection certificate =
15 N However, for up-gradation under ii) on basis of „Equipment Months‟
shall not be given before 15 months from date of issue of inspection
certificate for N/2th item.

Criteria whichever is earlier complied with from i) or ii) above can be
considered for up-gradation.

Note: Where minimum specified in service period specified is more than
mentioned in para (b) (i) above, the equipment months criteria shall be
change accordingly e.g.~if for an item minimum specified period is 18
months then equipment month will change by factor of 18/12 = 1.5 i.e.~12
N becomes 12X 1.5 = 18N and 15N becomes 15 X 1.2 = 22.5N.

N and T specified by directorate

\begin{enumerate}[<+->]
[a)]
\setcounter{enumi}{2}
\tightlist
\item
  Each Directorate must specify minimum quantity for each item in their
  directorate procedures for upgradation
\end{enumerate}

For certain critical \& safety items where it is not possible to
evaluate the performance, unless it is monitored upto next
overhaul/sufficient period, the directorate head shall prepare the
exception list specifying the period for such items and include it in
the directorate procedures. If, at any stage directorate need to change
the minimum quantity for upgradation, then proposal for same with
justification shall be send to Spl. DG/VD for approval.

\begin{enumerate}[<+->]
[a)]
\setcounter{enumi}{3}
\item
  mandatory requirement :Vendor should possess valid ISO 9001
  certificate for manufacture of same/similar item at his works address.
\item
  The name of the firm should appear in ``List of Vendors for
  Development Orders‟.
\item
  Service Performance - The performance of the firm should be
  satisfactory. The performance criteria for each item shall be decided
  by the Directorate head taking into account the criteria laid down by
  the concerned design directorate, like limiting FRPCPY as per the
  formula for calculating the FRPCPY of the item (where ever applicable)
  as under:
\end{enumerate}

FRPCPY = \{(No.~Of failures * 100\emph{12) /(Population}Period in
months)\}

The performance reports received through web portal of RDSO/through
official email (no manual feedback should be asked for) up to the date
of eligibility of the firm shall be taken as performance criteria for
considering up gradation.\\
Directorate Head shall take up the issue with concerned PHODs of zonal
Railways on regular basis. (Reference: - DG/RDSO‟s DO No.~13/vig/Policy
dated 13/11/2014 \& CVO/RDSO‟s Note No.~Comp/8.02.10 dated 31/03/2016)

\begin{enumerate}[<+->]
[a)]
\setcounter{enumi}{6}
\tightlist
\item
  In exceptional circumstances where waiver of laid down conditions for
  upgradation of firm is required, the same will require prior approval
  of Spl. DG/VD/RDSO.
\end{enumerate}

Note: Many equipment/items which have RDSO specifications but vendor
approval is not done by RDSO are procured by railways on their own.
Inclusion of such items as identified by Railway Board in RDSO approved
list shall be done by inviting EOI for vendor registration in which
existing firms can also participate.

However, after approval of existing firms in ``List of RDSO Vendors for
developmental order‟ posted on RDSO website, the mandatory period of one
year for upgradation of firm to ``List of Approved Vendors‟ can be
dispensed with in case firm has already supplied minimum quantity for
upgradation from ``List of RDSO Vendors for developmental order‟ posted
on RDSO website to ``List of Approved Vendors‟ and performance of
product supplied earlier was satisfactory

4.1.2 Time period for applying for upgradation The firms classified as
„List of RDSO Vendors for developmental order‟ posted on RDSO website on
the vendor list can apply maximum six months in advance of the date due
for upgradation. The applications which are received between three to
six months in advance of the due date for upgradation shall be
processed. However status shall be upgraded on the due date subject to
the condition that all laid down criteria has been complied with. If any
sample is under testing at RDSO, the result of the test shall be
considered.

Applications received less than three months in advance of the due date
of upgradation, shall be processed. However, delay if any in upgradation
with respect to due date, shall be attributable to the late receipt of
application.

4.1.3 Suo moto upgradation of the developmental vendors: ``On receipt of
information at RDSO by the concerned directorate that a Developmental
Vendor has fulfilled the upgradation criteria as stipulated at para
4.1.1 above, after completion of 5 years period from the date of
entering into vendor directory as Developmental vendor (without any
quantity restriction ), the Directorate will initiate, Suo moto the
process of upgradation of the developmental vendor as Approved Vendor
and informed the vendor to be in readiness for Bidding/Supplying larger
quantities as expected from an Approved Vendor for that item.

Letter conveying enlistment as developmental vendor to every new firm
henceforth will bear a validity of 05 years period. After lapse of 4
year period from the registration , firm has to provide details to RDSO
about efforts/ orders executed during the last 4 year period and based
on the merit of the case, concerned Directorate may extend validity of
the firm‟s status as Developmental source for further 02 years / may
consider for upgrading as approved source.

4.1.4: After seven-year period as Developmental Vendor status (without
any quantity restriction), if the firm fails to comply with the
stipulated upgradation requirements, the firm will be delisted from the
list of developmental vendors.

A one time relaxation period of 03 years will be given to all vendors
whose registration status is being affected or expected to be affected
in next 03 years from the date of issue of this amendment.
\end{block}
\end{frame}

\end{document}
