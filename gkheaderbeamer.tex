\usepackage{forloop} % for loop
%\usepackage{amsmath,amssymb}
\usepackage{graphicx}      % Allows including images
\usepackage{booktabs}      % 
\usepackage[final]{pdfpages}
\mode<presentation>{
  \usetheme{AnnArbor}\setbeamercovered{transparent}  }
%\usetheme{PaloAlto}\setbeamercovered{transparent}  }
%\usetheme{CambridgeUS}\setbeamercovered{transparent}  } %red plain
%\usetheme{Antibes}\setbeamercovered{transparent}  } 
%AnnArbor, PaloAlto, CambridgeUS
%\mode<presentation>{\usetheme{?}\setbeamercovered{transparent}  }
%\mode<presentation>{\usetheme{Berkeley}\setbeamercovered{transparent}  }
%\mode<presentation>{\usetheme{Warsaw}\setbeamercovered{transparent}  }
%\mode<presentation>{\usetheme{Antibes}\setbeamercovered{transparent}  }
%\mode<presentation>{\usetheme{Berkeley}\setbeamercovered{transparent}  }
%mytheamlist =AnnArbor   Antibes  Berkeley  CambridgeUS
%Copenhagen Madrid  Warsaw  JuanLesPins
%Goettingen   Hannover  Marburg  PaloAlto Marburg 

%\usecolortheme{beaver}
%\usecolortheme{albatross} %wolverine}
%albatross(blue), beaver(red),crane(yellow),whale(blue),wolverine(yellow)
%\usecolortheme{seagull} %for black and white grey
%\usecolortheme{spruce}  %light green

%\usetheme[options]{name list} %sameas \usepackage for style file 
% named beamerthemename.sty for each name in the name list.
%\usecolortheme[options]{name list} Same as \usetheme for colours%
  % % Color style files are named beamercolorthemename.sty.

%\setbeamercolor{normal text}{bg=red!20}
%\setbeamercolor{background canvas}{bg=}  %good idea bg background

\graphicspath{  {img/},  {./}  {/chp} }
%%%%%%%%%%%

\usepackage[many]{tcolorbox}
\usetikzlibrary{decorations.pathmorphing} % logo seal
\usetikzlibrary{shadows}
\usetikzlibrary{decorations.text} % for spiral text 
\definecolor{paper}{RGB}{239,227,157}

%%%%%%%%%%%%%%%%%%%%%%%%%%%%%%%%%%%%%%%%%%%%%%%%%%%%%%%%%%%%%%%%%%%%
  %\begin{frame}<handout:0> %n for n frames, 0 suppress frame in handout

\usepackage{lipsum}
\usepackage{verbatim}

\usepackage{chronosys}

\usepackage{graphics}
\usepackage{graphicx} % default in beamer
\usepackage{pdfpages}

\usetikzlibrary{arrows}
\usepackage{fancybox}  %unessential

\usepackage{caption}
%\usepackage{subcaption}
%\usepackage{subfigure}   %causing errors

%\usepackage{pgfplots}
%\pgfplotsset{compat=newest}% <-moves axis labels near ticklabels 


%\usepackage{pgfplots}
\usepackage{tikz}
\usetikzlibrary{shapes} %,snakes}
\usetikzlibrary{spy,calc}
%\usepackage{decorations}
\usetikzlibrary{decorations, decorations.text} %tried logo

%\usepackage{pgfornament}
%\usepackage[object=vectorian]{pgfornament}
% \usepackage{tikz, pgfornament, tikzrput}

\usepackage{overpic}      %put command under picture envoir
%\usepackage{tikzrput}    %rput under \begin{tikzpicture}


%\usepackage[hyperref]{beamerarticle}
\usepackage{hyperref}

%\usepackage{amsmath,amssymb}

%\usepackage{tikz,times}
%\usepackage[paperwidth=25cm,paperheight=22cm,left=1cm,top=1cm]{geometry}
\usetikzlibrary{mindmap,backgrounds}
%\pagestyle{empty}

\newtcolorbox{tornpage}{enhanced jigsaw, 
  breakable, % allow page breaks
  frame hidden, % hide the default frame
  overlay={%
    \draw [
      fill=paper, % fill paper
      draw=paper!30!black, % boundary colour
      decorate, % decoration
      decoration={random steps,segment length=5pt,amplitude=5pt},  %amplitude=2pt size of waviness, ,segment length=2pt
      drop shadow, % shadow
    ]
    (frame.north west)--(frame.north east)--		% top line
    (frame.north east)--(frame.south east)--		% right line
    (frame.south east)--(frame.south west)--		% bottom line
    (frame.south west)--(frame.north west);		    % left line
  },
  parbox=false,
}

%%%%%%%%%%%%%%%%%%%%%%%%%%%%%%%%%%%%%%%%%%%%%%%%%%%%%%%%%%%%%%%%%
\useinnertheme[shadow=true]{rounded}

\usepackage{etoolbox}

\setbeamercolor{block title}{use=structure,fg=structure.fg,bg=structure.fg!20!bg}
\setbeamercolor{block body}{parent=normal text,use=block title,bg=block title.bg!50!bg}

\setbeamercolor{block title example}{use=example text,fg=example text.fg,bg=example text.fg!20!bg}
\setbeamercolor{block body example}{parent=normal text,use=block title example,bg=block title example.bg!50!bg}

\addtobeamertemplate{proof begin}{%
  \setbeamercolor{block title}{fg=black,bg=red!50!white}
  \setbeamercolor{block body}{fg=red, bg=red!30!white}
}{}

\BeforeBeginEnvironment{theorem}{
  \setbeamercolor{block title}{fg=black,bg=orange!50!white}
  \setbeamercolor{block body}{fg=orange, bg=orange!30!white}
}
\AfterEndEnvironment{theorem}{
  \setbeamercolor{block title}{use=structure,fg=structure.fg,bg=structure.fg!20!bg}
  \setbeamercolor{block body}{parent=normal text,use=block title,bg=block title.bg!50!bg, fg=black}
}

\BeforeBeginEnvironment{definition}{%
  \setbeamercolor{block title}{fg=black,bg=pink!50!white}
  \setbeamercolor{block body}{fg=pink, bg=pink!30!white}
}
\AfterEndEnvironment{definition}{
  \setbeamercolor{block title}{use=structure,fg=structure.fg,bg=structure.fg!20!bg}
  \setbeamercolor{block body}{parent=normal text,use=block title,bg=block title.bg!50!bg, fg=black}
}

%%%%%%%%%%%%%%%%%%%%%%%%%%%%%%%%%%%%%%%%%%%%%%%%%%%%%%%%%%%%%%%%%%
\subject{RDSO RS Presentation } % for info catalog.
\title[Review of Vendor Process ]{Review of Vendor Process}

%\subtitle[QA Improvements]{System Improvements in QA Working }
\author[ PED/RS/RDSO ]{
  %	         \includegraphics[height=2cm]{img/IRlogored.png}\\ 
  All Vendor Dealing Directorates,\\     RDSO, Lucknow }
%\date[LKO:\today]{LKO :\today}     % or \date[GKevent2016]{STACS Conference, 2003}.
\date[\today]{Lucknow :\today} 
%\author[Gopal Kumar, otherP]{GopalKumar\inst{1} \and OtherPerson\inst{2}}
%\institute[Expotech]{\inst{1}GGM/Expotech,\\ Expotech Division   
  %\and 	
  %\inst{2}GM/Expotech,\\ Expotech Division  }    

\pgfdeclareimage[height=01.5cm, width=01.5cm]{mylogo}{img/RDSO_Logo1.jpg} %img/IRlogored.png} %logo
\logo{\pgfuseimage{mylogo}}

\titlegraphic{\pgfuseimage{mylogo}}  
%Set canvas global, better use second tikspicture Opacity
%\setbeamertemplate{background canvas}
%{\includegraphics[width=0.5\paperwidth,height=0.5\paperheight]{RITESlogo.pdf}}
\setbeamertemplate{backgroundcanvas}{\begin{tikzpicture}
  \node[opacity=.1]{\includegraphics[width=\paperwidth]{img/RDSO_Logo1.jpg}}; %IRLogored.png}};
\end{tikzpicture} }

# added gk
\usebackgroundtemplate{%
\tikz[overlay,remember picture] \node[opacity=0.3, at=(current page.center)] {
   \includegraphics[height=\paperheight,width=\paperwidth]{example-image-a}};
}

%%%%%%%%%%%%%%%%
  %\setbeamercolor{frametitle}{bg={}}   % transparent background bg={}; use {} to limit \setbeamercolor and \usebackgroundtemplate  
  %%%%%%%%%%%%%%%
  
  
  % toc to pop up at each section:
    %\AtBeginSection[]{\begin{frame}<beamer>{Outline} \tableofcontents[currentsection] \end{frame}   } % toc at each Subsections
  %\AtBeginSubsection[]{ \begin{frame}<beamer>{Outline} \tableofcontents[currentsection,currentsubsection] \end{frame}   }
  
  %%\beamerdefaultoverlayspecification{<+->}  %uncover all step-wise
  
  %%%%%%%%%%%%%%%%%%%%%%%%%%%%%%%%%%%%%%%%%%%%%%%%%%%%%%%%%%%%%%%%%
  %%use \lecture with options: \AtBeginLecture, \includeonlylecturelecture label etc
  %% use with hyperlink additional slides \appendix<mode specification>
    %% use \againframe<overlay spec>[<def ovlay specs>][options]{name} resume!!
    %%%%%%%%%%%%%%%%%%%%%%%%%%%%%%%%%%%%%%%%%%%%%%%%%%%%%%%%%%%%%%%%%
  
  \setbeamertemplate{background canvas}{ %
    \begin{tikzpicture}\node[opacity=.05]{
      \includegraphics[width=\paperwidth]{IRlogored.png}};
    \end{tikzpicture} }
  %\setbeamertemplate{background}{\includegraphics[width=\paperwidth]{example-image.pdf}}
  %\setbeamercolor{frametitle}{bg={}}   % transparent background bg={}; use {} to limit \setbeamercolor and \usebackgroundtemplate  
  
  %%%%%%%%%%%Define box and box title style
  \tikzstyle{mybox}      =[draw=red,fill=blue!20,very thick,rectangle, rounded  corners, inner sep=10pt, inner ysep=20pt]
  \tikzstyle{fancytitle}  =[fill=red, text=white]
  
  %%%%%%%%%%%%%%%%%%%%%%%%%%%%%%%%%%%%%%%%%%%%%%%%%%%%%%%%%%%%%%%%%%%%%
  
  \usetikzlibrary{shadows}
  \usepackage{lipsum}
  
  \usepackage{xparse}
  %\usepackage{punk}  %for punky fonts!
    
    \definecolor{myyellow}{RGB}{242,226,149}
  
  %%%%%%%%%%%%%%%%%%%%%%%%%%%%%%%%%%%%%%%%%%%%%%%%%%%%%%%%%%%%%%%%%%%
  \NewDocumentCommand\StickyNote{O{2.5cm}mO{3.5cm}}{
    \begin{tikzpicture}
    \node[drop shadow={shadow xshift=2pt,shadow yshift=-4pt},
          inner xsep=7pt,inner ysep=10pt,fill=myyellow,xslant=-0.1,yslant=0.1]{\parbox[t][#1][c]{#3}{#2}};
            \end{tikzpicture}   }
    
    %\pgfplotset{compat= 1.14}
    %%%%%%%%%%%%%%%%%%%%%%%%%%%%%%%%%%%%%%%%%%%%%%%%%%%%%%%%%%%%%%%%%%%%%
    